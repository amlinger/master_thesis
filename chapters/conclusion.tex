%%%%%%%%%%%%%%%%%%%%%%%%%%%%%%%%%%%%%%%%%%%% 80 line marker %%%%%%%%%%%%%%%
Clustering and classification of data are two closely related subjects, 
that can utilize methodologies that diverge more or less. In this thesis 
various clustering methods consisting of representatives from different
algorithm families have been evaluated based on their suitability for 
life-logging devices, and in particular, the Narrative Clip. Bayesian 
inference has then been used for classification of the activities detected 
by the clustering algorithms, allowing labels to be attached to each 
activity. 

%CLUSTERING
Comparing produced clusters and expecting exact similarity between different
algorithms is not a good approach. As mentioned early in this report, 
there is no universal definition of a cluster and therefore no universal 
clustering algorithm. Taking this into account, it is a better approach to
venture beyond strict similarity in produced clustering, and compare other 
metrics, such as number of detected cluster, performance, Silhouettes and so
on. But the most important examination point is the cluster algorithms 
characteristics and abilities, basically what it was meant to do.

CLARANS was the representative from the category of partitioning clustering 
methods. While partitioning methods in general are easy to implement and to 
understand, they have drawbacks that are significant, embodied by the results 
of CLARANS. The performance is poor due to the knowledge about the number of 
clusters being a prerequisite, requiring several runs of the algorithm and 
evaluating the result. Choosing the clustering parameters for CLARANS is 
unintuitive, since it is based on random search of a solution graph, and may 
find different solutions to the same problem, with varying quality. Bearing 
all this in mind, partitioning methods and particularly CLARANS seem 
inappropriate for life-logging applications, as the knowledge of the number
of clusters that exist on beforehand is a luxury that we usually do not have.
The execution time necessary to find the appropriate number of clusters
make usage of this not feasible, as it is implied by being automatic, that
this should be used in real-time.

DBSCAN performs well in run time, both for small and for large sets of data 
points. The choice of clustering parameters affects the amount of clusters, 
and the type of clusters one is willing to find. In the tested configuration, 
the algorithm seems most suitable for finding a little bit more fuzzy clusters
containing a rough position and outline, but not necessarily sorted by time. 
This is useful in activity detection, where a more neatly presented and 
approximated cluster is more appreciated and easier to use for a 
classification algorithm.

SLINK runs slightly faster than DBSCAN in the considered test cases, and 
is more flexible in the way it produces clusters. This allows for clusters to 
be cutoffs in a time-line, which is suitable for Moment detection, where a more 
strict definition of a cluster is required. The same could be said for several 
other applications, where time or other factors need to be sorted in first hand 
to produce clusters. 

This yields two different areas where DBSCAN and SLINK are applicable, in
activity detection and in Moment division, respectively. Life-logging is in 
its essence full of data mining problems that needs to be solved, and utilizing
these or other algorithms on other problems is very likely, but examining the
entire life-logging area is not within the scope of this article. The conclusion
is just that these two applications is appropriate for these two problems.

Silhouettes are not a simple metric to use when evaluating different types of
clustering algorithm, due to the different definitions of a cluster and the
varying output from the algorithm. Although not being able to simply compare
numbers in order of determining quality of a clustering algorithms produced
output, Silhouettes could still be useful to examine how an algorithm behaves,
and why the produced result look the way they do. 

% CLASSIFICATION

When it comes to classification, the results seem somewhat individualized, 
both when it comes to what users think is important regarding assigned classes, 
and especially the model parameters determining the classes. 

From a simple model promising results were achieved, when allowing an iterative 
approach in a Bayesian manner. After all, there is nothing more Bayesian than 
updating beliefs after witnessing evidence! The used model was made simple to 
make it easy to perform an update step, thus showing that this improved the 
quality of the output significantly. 

Model changes, even seemingly small, are more powerful than Bayesian learning 
via learning data and model fitting. In this thesis, a rather small set of
learning data was used, due to both time and resource constraints. This caused
parameters not to be set too robustly, especially parameters with great 
variance suffered by not being able to converge fast enough. From this, we 
learn that studying the parameter values is important as well after learning
to be confident in a model.

\section{Further Research}
This report presents an overview for clustering and classification, and 
discusses suitability of some methods as well as identifying pitfalls. 
Life-logging in its essence is about in an automated fashion storing 
activities for the monitored object, in Narrative's case, the end user. 
This usually results in vast amounts of data that needs to be processed 
in order of retrieving summaries or more qualitative data. Everything 
that automates and improves the process of automation is interesting to
this area, as it improves the quality of service. 

Investigating model learning and Bayesian network learning would be an 
interesting next step for taking the automated process one step further.

\subsubsection{Phone coupling}
As of now, the GPS data obtained by the Clip is rather sporadic, and
not very accurate. Phones today are 
however better at positioning, and utilizes other methods such as 
WiFi for a better positioning approximation \cite{iphone-wifi}.

It does also seem likely that a user carries his or her phone while 
wearing the Clip, and making use of this data should lead to a 
more quantitative and qualitative positional data, in its turn 
simplifying and making the result of clustering more useful to the
end users. Examining what more quantitative and qualitative data 
produces from a starting point is appealing.

\subsubsection{Momentification}
Introducing other clustering factors than geo-spatial data and 
time labels seem feasible as well, and especially monitoring 
statistical breakpoints in sensor-data proves especially promising. 
To introduce this either as another dimension in the clustering data, 
or correlation the detected geo-spatial clusters with the proposed 
clusters based on statistical breakpoints.

\subsubsection{Dynamic Learning}
An interesting approach for extending the classification would be to
introduce dynamic learning for the classification algorithm. As is 
stated previously, the model parameters seem somewhat individualized, 
and making these depend on user feedback. Such feedback can both 
come in the form of explicit evaluations on a web-based service, or 
by observing user behaviour and assessing whether certain actions are
a positive or a negative stimuli.